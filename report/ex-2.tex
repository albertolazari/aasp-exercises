$S \vdash_{E_{enc}} s$ is solved by the following tree:

\begin{prooftree}
        \AxiomC{senc($s, \langle n_A, n_B \rangle$)}
                \AxiomC{\vdots}
            \UnaryInfC{$n_A$}
                    \AxiomC{$\mathsf{senc}(\mathsf{aenc}(n_B, \mathsf{pk}(sk_A)), n_A)$}
                        \AxiomC{\vdots}
                    \UnaryInfC{$n_A$}
                \BinaryInfC{$\mathsf{sdec}(\mathsf{senc}(\mathsf{aenc}(n_B, \mathsf{pk}(sk_A)), n_A), n_A)$}
                \RightLabel{$=_{E_{enc}}$}
                \UnaryInfC{$\mathsf{aenc}(n_B, \mathsf{pk}(sk_A))$}
                \AxiomC{$sk_A$}
            \BinaryInfC{$\mathsf{adec}(\mathsf{aenc}(n_B, \mathsf{pk}(sk_A)), sk_A)$}
            \RightLabel{$=_{E_{enc}}$}
            \UnaryInfC{$n_B$}
        \BinaryInfC{$\langle n_A, n_B \rangle$}
    \BinaryInfC{$\mathsf{sdec}(\mathsf{senc}(s, \langle n_A, n_B \rangle), \langle n_A, n_B \rangle)$}
    \RightLabel{$=_{E_{enc}}$}
    \UnaryInfC{$s$}
\end{prooftree}

\noindent
Given $S \vdash_{E_{enc}} n_A$, which is solved by the following proof tree:

\begin{prooftree}
    \AxiomC{$\mathsf{aenc}(n_A, \mathsf{pk}(sk_B))$} \AxiomC{$sk_B$}
    \BinaryInfC{$\mathsf{adec}(\mathsf{aenc}(n_A, \mathsf{pk}(sk_B)), sk_B)$}
    \RightLabel{$=_{E_{enc}}$}
    \UnaryInfC{$n_A$}
\end{prooftree}

\noindent
Note that the rules without a label are not the ones defined in $\mathcal{I}_{DY}$ (though they are the same). They are derived from the functions, generated from the equivalence steps. The only exception is:
\begin{prooftree}
    \AxiomC{$a$} \AxiomC{$b$}
    \BinaryInfC{$\langle a, b \rangle$}
\end{prooftree}
but $\langle a, b \rangle$ could be defined as a 2-ary function $\mathsf{pair}(a, b)$.
