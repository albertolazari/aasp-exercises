The following code models the protocol in ProVerif:

\texttt{\lstinputlisting{../protocol.pv}}

\subsection{Authentication}
Authentication is proved by the first query: the event occurs only when $B$ recieves $m$ from $A$. \\
$B$ can be certain that $m$ comes from $A$, because it decrypts it from a message encrypted with $N_b$, that could only be decrypted by $A$, because $B$ encrypted asymmetrically with $pk_A$, thus requiring $sk_A$ to decrypt. \\
The event is reached beacuse ProVerif outputs the following statement:
$$
\texttt{Query not event(authentication(m\_2,nb\_2)) is false.}
$$
which means that the event is reachable.

\subsection{Integrity}
Integrity is proved by the second query: the event occurs only if $A$ recieves back from $B$ the hash of $m$. $A$ checks wether the hash sent from $B$ and the real $\mathsf{h}(m)$ are the same. \\
The event is reached beacuse ProVerif outputs the following statement:
$$
\texttt{Query not event(integrity(m\_2)) is false.}
$$
which means that the event is reachable.

\subsection{Confidentiality}
Confidentiality is proved by the last query: it checks wether an external attacker can deduce the message $m$, but since it cannot only $A$ and $B$ know the message. \\
The last query is true because ProVerif outputs the following statement:
$$
\texttt{Query not attacker(m[x = v,!1 = v\_1]) is true.}
$$
