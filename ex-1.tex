\subsection{Inference system}
    $\mathcal{I}_{DY}$ provides the following inference rules:

    \begin{prooftree}
        \AxiomC{$x$} \AxiomC{$y$}
        \LeftLabel{(senc)}
        \BinaryInfC{senc($x, y$)}

        \AxiomC{$x$} \AxiomC{$y$}
        \LeftLabel{(aenc)}
        \BinaryInfC{aenc($x, y$)}

        \noLine
        \BinaryInfC{}
    \end{prooftree}

    \begin{prooftree}
        \AxiomC{$x$} \AxiomC{$y$}
        \LeftLabel{(pair)}
        \BinaryInfC{$\langle x, y \rangle$}

        \AxiomC{$\langle x, y \rangle$}
        \LeftLabel{($\mathrm{pair_1}$)}
        \UnaryInfC{$x$}

        \AxiomC{$\langle x, y \rangle$}
        \LeftLabel{($\mathrm{pair_2}$)}
        \UnaryInfC{$y$}

        \noLine
        \TrinaryInfC{}
    \end{prooftree}

    \begin{prooftree}
        \AxiomC{senc($x, y$)} \AxiomC{$y$}
        \LeftLabel{(sdec)}
        \BinaryInfC{$x$}

        \AxiomC{aenc($x, \mathrm{pk}(y))$} \AxiomC{$y$}
        \LeftLabel{(adec)}
        \BinaryInfC{$x$}

        \noLine
        \BinaryInfC{}
    \end{prooftree}

\subsection{Proof tree solving}
    Let $S =
        sk_A,
        sk_B,
        \mathrm{aenc}(n_A, \mathrm{pk}(sk_B)),
        \mathrm{senc}(\mathrm{aenc}(n_B, \mathrm{pk}(sk_A)), n_A),
        \mathrm{senc}(s, \langle n_A, n_B \rangle)
    $. \\
    $S \vdash_{\mathcal{I}_{DY}} s$ is solved by the following proof tree:

    \begin{prooftree}
        \AxiomC{senc($s, \langle n_A, n_B \rangle$)}
                \AxiomC{\vdots}
            \UnaryInfC{$n_A$}
                    \AxiomC{senc($\mathrm{aenc}(n_B, \mathrm{pk}(sk_A)), n_A$)}
                        \AxiomC{\vdots}
                    \UnaryInfC{$n_A$}
                    \RightLabel{sdec}
                \BinaryInfC{aenc($n_B, \mathrm{pk}(sk_A)$)}
                \AxiomC{$sk_A$}
                \RightLabel{adec}
            \BinaryInfC{$n_B$}
            \RightLabel{pair}
        \BinaryInfC{$\langle n_A, n_B \rangle$}
        \RightLabel{sdec}
        \BinaryInfC{$s$}
    \end{prooftree}

    \noindent
    Given $S \vdash_{\mathcal{I}_{DY}} n_A$, which is solved by the following proof tree:

    \begin{prooftree}
        \AxiomC{aenc($n_A, \mathrm{pk}(sk_B)$)} \AxiomC{$sk_B$}
        \RightLabel{adec}
        \BinaryInfC{$n_A$}
    \end{prooftree}

\subsection{Local theory}
$\mathcal{I}_{DY}$ is a \emph{local theory}, because, given $S$ and $t$, every label in the proof tree that solves $S \vdash_{\mathcal{I}_{DY}} t$ is a subterm of $S \cup \{t\}$, i.e. $\mathcal{I}_{DY}$ does not introduce new terms (other than $t$) that are not already present in S.
